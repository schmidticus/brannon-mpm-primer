%% LyX 2.0.4 created this file.  For more info, see http://www.lyx.org/.
%% Do not edit unless you really know what you are doing.
\documentclass[english]{article}
\usepackage[T1]{fontenc}
\usepackage[latin9]{inputenc}
\usepackage{amssymb}
\usepackage{babel}
\begin{document}
\begin{center}
{\Large Introduction to the Material Point Method}
\par\end{center}{\Large \par}

\begin{center}
{\Large with review of the Finite Element Method}
\par\end{center}{\Large \par}

~

\begin{center}
{\large Steven Schmidt}
\par\end{center}{\large \par}

~

~

This is an attempt to describe in as brief, plain, and specific terms
as possible what the Material Point Method is, and how it works, in
relation to the Finite Element method from which it was derived. As
part of this, the Finite Element method will be reviewed.

~

\textbf{Finite Element Method}

~

In principle, the Finite Element method is a method of finding an
approximate solution to a Partial or Ordinary Differential Equation
(ODE or PDE). There are many ODEs and PDEs that can be solved using
this method. Here are some examples:

The Heat Equation:

\[
-\triangledown^{2}u(x,y)=\frac{\partial u(x,y)}{\partial t}
\]


The Navier-Stokes Equations:

\[
(etc.)
\]


\[
\]


In the Engineering world, one in particular is very commonly solved
using this method, which is:

\[
\]


\[
\]

\end{document}
