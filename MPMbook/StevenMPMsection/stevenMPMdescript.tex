%% LyX 2.0.4 created this file.  For more info, see http://www.lyx.org/.
%% Do not edit unless you really know what you are doing.
\documentclass[english]{article}
\usepackage[T1]{fontenc}
\usepackage[latin9]{inputenc}
\usepackage{amsbsy}
\usepackage{amssymb}
\usepackage{esint}
\usepackage{babel}
\begin{document}
\begin{center}
{\Large Introduction to the Material Point Method}
\par\end{center}{\Large \par}

\begin{center}
{\Large with review of the Finite Element Method}
\par\end{center}{\Large \par}

~

\begin{center}
{\large Steven Schmidt}
\par\end{center}{\large \par}

~

~

This is an attempt to describe in as brief, plain, and specific terms
as possible what the Material Point Method is, and how it works, in
relation to the Finite Element method from which it was derived. As
part of this, the Finite Element method will be reviewed.

~

\textbf{Finite Element Method}

~

In principle, the Finite Element method is a method of finding an
approximate solution to a Partial or Ordinary Differential Equation
(ODE or PDE). There are many ODEs and PDEs that can be solved using
this method. Here are some examples:

The Heat Equation:

\[
-\triangledown^{2}u(x,y)=\frac{\partial u(x,y)}{\partial t}
\]


The Navier-Stokes Equations:

\[
(etc.)
\]


\[
(etc.)
\]


\[
\]


In the Engineering world, one in particular is very commonly solved
using this method, which is: 

~

Static Stress/Strain Equation (where ``$u$'' represents displacement):

\[
\frac{\partial}{\partial x}\left(AE\frac{\partial u}{\partial x}\right)+f(x)=0
\]
Dynamic Stress/Strain Equation:

\[
\frac{\partial}{\partial x}\left(AE\frac{\partial u(x,t)}{\partial x}\right)+f(x)=\rho(x)A(x)\frac{\partial^{2}u}{\partial t^{2}}
\]
(Note that while I wrote these in their 1D form, they can be represented
in n-D.)

~

Some important notes:

~

1. The Stress/Strain equation is what is very frequently solved when
doing FEM and MPM, so it is the equation that we will examine in this
document. (Be aware that often times when people say ``Finite Element''
they mean using FEM to solve this equation).

2. However, I want to emphasize that the FEM and MPM methods are at
their most fundimental level simply a means of solving any ODE or
PDE. The set of values we obtain at the end of the FEM or MPM method
is used to obtain an approximate representation of ``$u(x)$'' (or
``$u(x,y,...)$'') at each timestep!

3. The process we go through to obtain the FEM and MPM algorithms
results in a representation of these equations that are very different
in appearance to the original equations.

4. The process we describe of solving a FEM or MPM method often requires
constructing large matrices and solving large systems of equations,
but in many cases the actual computer algorithm will not construct
these matrices, but rather do the equivalent through other iterative
or algorithmic means. You might ask yourself, ``Wait, what happened
to the \_\_\_\_\_\_ matrix? I don't see it anywhere in this algorithm!''
Yup, you don't see it anywhere in that algorithm. But the mathematical
equivalent is there... somewhere.

~

\textbf{Review of the standard Strong Form to Weak Form to Discretization
of the Finite Element Method}

~

~

\textbf{Strong Form to Weak Form:}

~

First, we have:

\[
\frac{d}{dx}[A(x)E(x)\frac{du(x,t)}{dx}]+f(x,t)=\rho(x)A(x)a(x,t)
\]


(With units:)

\[
(\frac{1}{m})\left((m^{2})(\frac{kg\cdot m/s^{2}}{m^{2}})(\frac{m}{m})\right)+\frac{(kg\cdot m/s^{2})}{m}=(\frac{kg}{m^{3}})(m^{2})(\frac{m}{s^{2}})
\]


(simplifying:)

\[
\left((\frac{kg\cdot m/s^{2}}{m})\right)+\frac{(kg\cdot m/s^{2})}{m}=\frac{(kg\cdot m/s^{2})}{m}
\]


$u(x,t)$ is displacement. {[}$m${]}

$a(x,t)$ is the acceleration. Put another way, $a(x,t)=\frac{d^{2}u(x,t)}{dt^{2}}$.
{[}$m/s^{2}${]}

$f(x,t)$ is the ``body force'' (really a ``force density'') (usually
something like gravity) {[}$N/m${]}

$A(x)$ is the cross sectional area of a bar. (Time invariant). {[}$m^{2}${]}

$E(x)$ is the Young's Modulous of the bar. (Time invariant). {[}$N/m^{2}${]}

$\rho(x)$ is a volume-mass-density (mass per unit volume), which
makes $\rho(x)A(x)$ a linear-mass-density (mass per unit length).
(Time invariant). {[}$kg/m^{3}]$

~

Continuing, we integrate everywhere over the domain and multiply by
a weight function $w(x,t)$:

\[
\int_{0}^{L}w(x,t)\frac{d}{dx}[A(x)E(x)\frac{du(x,t)}{dx}]dx+\int_{0}^{L}w(x,t)f(x,t)dx=\int_{0}^{L}w(x,t)\rho(x)A(x)a(x,t)dx
\]


Integration by parts to move the derivative in x to the weight function:

\[
w(x,t)A(x)E(x)\frac{du(x,t)}{dx}\vert_{x=0}^{x=L}-\int_{0}^{L}\frac{dw(x,t)}{dx}A(x)E(x)\frac{du(x,t)}{dx}dx+\int_{0}^{L}w(x,t)f(x,t)dx=\int_{0}^{L}w(x,t)\rho(x)A(x)a(x,t)dx
\]


\[
w(L,t)A(L)E(L)\frac{du(L,t)}{dx}-w(0,t)A(0)E(0)\frac{du(0,t)}{dx}-\int_{0}^{L}\frac{dw(x,t)}{dx}A(x)E(x)\frac{du(x,t)}{dx}dx+\int_{0}^{L}w(x,t)f(x,t)dx=\int_{0}^{L}w(x,t)\rho(x)A(x)a(x,t)dx
\]


For sake of simplicity, the boundary terms go to zero: (TODO: find
out why).

\[
-\int_{0}^{L}\frac{dw(x,t)}{dx}A(x)E(x)\frac{du(x,t)}{dx}dx+\int_{0}^{L}w(x,t)f(x,t)dx=\int_{0}^{L}w(x,t)\rho(x)A(x)a(x,t)dx
\]


~

\textbf{Discretization:}

~

$\boldsymbol{N}$ is the vector of shape functions.

$\boldsymbol{B}$ is the vector of the derivative of the shape functions.

$\boldsymbol{w(t)}$ is the vector of values of $w(x,t)$ at $x=x_{i}$,
the location of each node $i$.

$\boldsymbol{u(t)}$ is the vector of values of $u(x,t)$ at $x=x_{i}$,
the location of each node $i$.

$\boldsymbol{a(t)}$ is the vector of values of $a(x,t)$ at $x=x_{i}$,
the location of each node $i$.

~

\[
w(x,t)\approx\boldsymbol{w(t)}^{T}\boldsymbol{N}=\boldsymbol{N}^{T}\boldsymbol{w(t)}
\]


\[
\frac{dw(x,t)}{dx}\approx\boldsymbol{w(t)}^{T}\boldsymbol{B}=\boldsymbol{B}^{T}\boldsymbol{w(t)}
\]


\[
\frac{du(x,t)}{dx}\approx\boldsymbol{u(t)}^{T}\boldsymbol{B}=\boldsymbol{B}^{T}\boldsymbol{u(t)}
\]


\[
a(x,t)\approx\boldsymbol{a(t)}^{T}\boldsymbol{N}=\boldsymbol{N}^{T}\boldsymbol{a(t)}
\]


Thus we have:

\[
-\int_{0}^{L}\boldsymbol{w(t)}^{T}\boldsymbol{B}A(x)E(x)\boldsymbol{B}^{T}\boldsymbol{u(t)}dx+\int_{0}^{L}\boldsymbol{w(t)}^{T}\boldsymbol{N}f(x,t)dx=\int_{0}^{L}\boldsymbol{w(t)}^{T}\boldsymbol{N}\rho(x)A(x)\boldsymbol{N}^{T}\boldsymbol{a(t)}dx
\]


We can cancel out the $\boldsymbol{w(t)}^{T}$:

\[
-\int_{0}^{L}\boldsymbol{B}A(x)E(x)\boldsymbol{B}^{T}\boldsymbol{u(t)}dx+\int_{0}^{L}\boldsymbol{N}f(x,t)dx=\int_{0}^{L}\boldsymbol{N}\rho(x)A(x)\boldsymbol{N}^{T}\boldsymbol{a(t)}dx
\]


So now we construct a linear system:

\[
-\boldsymbol{K}\boldsymbol{u}+\boldsymbol{f}=\boldsymbol{M}\boldsymbol{a}
\]


where

$\boldsymbol{K}$ is the stiffness-matrix.

$\boldsymbol{u(t)}$ is the nodal displacement vector.

$\boldsymbol{f}$ is the body-force vector.

$\boldsymbol{M}$ is the mass matrix.

$\boldsymbol{a(t)}$ is the acceleration vector.

Note: $\boldsymbol{v(t)}$, the nodal velocity vector, is also known
at the beginning of the timestep.

~

\[
\boldsymbol{K}=\int_{0}^{L}A(x)E(x)\boldsymbol{B}\boldsymbol{B}^{T}dx
\]


\[
\boldsymbol{f}=\int_{0}^{L}\boldsymbol{N}f(x,t)dx
\]


\[
\boldsymbol{M}=\int_{0}^{L}\rho(x)A(x)\boldsymbol{N}\boldsymbol{N}^{T}dx
\]


So we have:

\[
\left\{ -\left[\int_{0}^{L}A(x)E(x)\boldsymbol{B}\boldsymbol{B}^{T}dx\right]\left\{ \boldsymbol{u(t^{n})}\right\} +\left\{ \boldsymbol{f(t^{n})}\right\} \right\} =\left[\int_{0}^{L}\rho(x)A(x)\boldsymbol{N}\boldsymbol{N}^{T}dx\right]\left\{ \boldsymbol{a(t^{n})}\right\} 
\]


~

\textbf{Important Note:}

It is here that we have two directions we can go. In the Static state,
the accelerations are zero, so the right hand side goes to zero, and
we have:

\[
-\left[\int_{0}^{L}A(x)E(x)\boldsymbol{B}\boldsymbol{B}^{T}dx\right]\left\{ \boldsymbol{u(t^{n})}\right\} +\left\{ \boldsymbol{f(t^{n})}\right\} =0
\]


In this case, it usually is the displacement that is the unknown,
and so we arange our problem into a linear system to find the displacement:

\[
\left[\int_{0}^{L}A(x)E(x)\boldsymbol{B}\boldsymbol{B}^{T}dx\right]\left\{ \boldsymbol{u(t^{n})}\right\} =\left\{ \boldsymbol{f(t^{n})}\right\} 
\]


\[
\left\{ \boldsymbol{u(t^{n})}\right\} =\left[\int_{0}^{L}A(x)E(x)\boldsymbol{B}\boldsymbol{B}^{T}dx\right]^{-1}\left\{ \boldsymbol{f(t^{n})}\right\} 
\]


However, when the acceleration is non-zero, we usually flip the problem
around, and assume that the displacement is known, and that it's the
acceleration that is unknown. The acceleration is solved for so that
we can calculate the changes to the displacement in time, to form
the initial conditions of the following timestep:

\[
\left\{ \boldsymbol{a(t^{n})}\right\} =\left[\int_{0}^{L}\rho(x)A(x)\boldsymbol{N}\boldsymbol{N}^{T}dx\right]^{-1}\left\{ -\left[\int_{0}^{L}A(x)E(x)\boldsymbol{B}\boldsymbol{B}^{T}dx\right]\left\{ \boldsymbol{u(t^{n})}\right\} +\left\{ \boldsymbol{f(t^{n})}\right\} \right\} 
\]


\textbf{Important Note:}

This is one of those times where the formal mathematical notation
looks a lot different than the computer-code. When this time-variant
equation is programmed into a computer, it is very rare to actually
construct the global matrices in order to solve for the acceleration.
It is more typical, rather, to iterate over each element, calculating
for each the force it exerts on the nodes it touches, and then summing
on each node all the forces acting on that node. Then the code forward-integrates
in time to find the new location (or displacement) on each node.

\[
\]
\[
\]


\[
\]

\end{document}
