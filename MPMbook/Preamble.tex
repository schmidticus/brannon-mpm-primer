%\documentclass[12pt,reqno,titlepage]{book}
\documentclass[fleqn]{book}
%\usepackage{fullpage}%gives 1inch margins
\usepackage{amsmath}
\usepackage{amsthm}
\usepackage{bm}%to get bold math symbols
\usepackage[pdftex]{graphicx}
\usepackage{verbatim}
\usepackage[margin=2.5cm]{geometry}%for the margins headers footers etc
\usepackage{hyperref}%for the hyperlinks-COOL and must use
\usepackage[tt]{titlepic}%package i downloaded to put a pic in the titlepage
\usepackage{helvet}
\usepackage[usenames,dvipsnames]{color}%options to use names like redviolet and others
\usepackage{xspace}
\usepackage[mathscr]{eucal}%Defines which font to use with \mathscr
\usepackage{amssymb}%needed for \mathbb
\usepackage{enumerate}
\usepackage{comment}

%This lets you define your own style for showing the date. Below, we are defining the international standard format for dates.
\usepackage[yyyymmdd,hhmmss]{datetime}
\newdateformat{mydate}{\THEYEAR-\twodigit{\THEMONTH}-\twodigit{\THEDAY}}
\usepackage{graphicx}
%WARNING: If we instead use package ``graphics'', then this file would give an error
% Paragraph ended before \Gin@iii was complete
\graphicspath{{Figs/}}%use this to tell LaTeX where to look for images.
\usepackage{rotating} %this is what gives landscape figures

\makeatletter
\newcommand\ackname{Acknowledgements}
\if@titlepage
  \newenvironment{acknowledgements}{%
      \titlepage
      \null\vfil
      \@beginparpenalty\@lowpenalty
      \begin{center}%
        \bfseries \ackname
        \@endparpenalty\@M
      \end{center}}%
     {\par\vfil\null\endtitlepage}
\else
  \newenvironment{acknowledgements}{%
      \if@twocolumn
        \section*{\abstractname}%
      \else
        \small
        \begin{center}%
          {\bfseries \ackname\vspace{-.5em}\vspace{\z@}}%
        \end{center}%
        \quotation
      \fi}
      {\if@twocolumn\else\endquotation\fi}
\fi
\makeatother


%%%%%%%%%%%%%%%%%%%%%%%%%%%%%%%%%%%%%%%%%%%5
%\usepackage[hidelinks]{hyperref} %This is simple way to make links black. Below shows how to do it with greater control...
\usepackage{hyperref}

\definecolor{mycitecolor}{rgb}{0.4,0.15,0.15}
\definecolor{mylinkcolor}{rgb}{0.15,0.15,0.4}%dark-blue
\definecolor{myurlcolor}{rgb}{0,0,0.5}
\hypersetup{
  colorlinks   = true, %Colours links instead of ugly red boxes for cross-references and hyperlinks
  urlcolor     = myurlcolor, %Colour for external hyperlinks
  linkcolor    = mylinkcolor, %Colour of internal links
  citecolor   = mycitecolor %Colour of citations
}
\urlstyle{rm} %so it doesn't use a typewriter font for URLs.

%\url{http://ihome.ust.hk/~tanjim/verylongaddresslikethisone-111111zxzxzxzxzxzxzx
%zxzxsqut_high.pdf}
%%%%%%%%%%%%%%%%%%%%%%%%%%%%%%%%%%%%%%%%%%%5


%%%%%%%%%%%%%%%%%%%%%%%%%%%%%%%%%%%%%%%%%%%%%%%%%%%%%%% Local macros

%@@@@@@@@@@@@@@@@@@@@@@@@@@@@@
\newenvironment{MyAbstract}%
{
\leftskip=1in
\rightskip=0.5in
}
{}
%@@@@@@@@@@@@@@@@@@@@@@@@@@@@@




%%%%%%%%%%%%%%%%%%%%%%%%%%%%%%%%%%%%%%%%%%%%%%%%%%%%%%%%  From rmb_crossref.sty
\newcommand{\fig}[1]{Fig.~\ref{#1}}              % Figure
\newcommand{\eqn}[1]{ Eq.~(\ref{#1})\xspace}
\newcommand{\eqs}[2]{ Eqs.~(\ref{#1}) and (\ref{#2}) }
\newcommand{\Equation}[1]{Equation~(\ref{#1})}   % Equation at beg. of sentence.
%%% Usage: \showone{picture-basename}{width-fraction}
\newcommand{\showone}[2]
{%
    \begin{center}
        \includegraphics[width=#2\textwidth]{#1}%
    \end{center}
}
\newcommand{\sect}[1]{Section~\ref{#1}}          % Section

%%%%%%%%%%%%%%%%%%%%%%%%%%%%%%%%%%%%%%%%%%%%%%%%%%%%%%%% From rmb_draft.sty
\definecolor{authorNoteColor}{rgb}{.8,0,0}
\newcommand{\AuthorNote}[1]{{\color{authorNoteColor} \sffamily{\textbf{#1}}}}


%%%%%%%%%%%%%%%%%%%%%%%%%%%%%%%%%%%%%%%%%%%%%%%%%%%%%%%% From rmb_mathTypesetting.sty
\newcommand{\mathscript}[1]{%
    \ensuremath{\mathscr{#1}}%
}%
\newcommand{\scriptl}{\ell}
\newcommand{\cth}{\ensuremath{c^\text{th}}\xspace}
\newcommand{\eeth}{\ensuremath{e^\text{th}}\xspace}
\newcommand{\ith}{\ensuremath{i^\text{th}}\xspace}
\newcommand{\gth}{\ensuremath{g^\text{th}}\xspace}
\newcommand{\pth}{\ensuremath{p^\text{th}}\xspace}
\newcommand{\nth}{\ensuremath{n^\text{th}}\xspace}
\newcommand{\Vector}[1]{\ensuremath{\bm{#1}}\xspace}%
\newcommand{\Scalar}[1]{{\ensuremath{#1}}\xspace}%
\newcommand{\definedEqual}{:=}%
\newcommand{\boxit}[1]{\ensuremath{\boxed{\ensuremath{#1}}}}%
\newcommand{\by}{\ensuremath{\times}}  %The box was dimensioned 2\by3\by4
                                       %Alternative with siunitx package:
                                       %\num{2 x 3 x 4}

% usage:   \Derivpp{y}{x} will give dy/dx  with d's partials
\newcommand{\Derivpp}[2]{
        \ensuremath{\frac{\partial {#1}}{\partial {#2}}}
}

% usage:   \Derivppp{y}{x}{z} will give (dy/dx)_z partial deriv showing const
\newcommand{\Derivppp}[3]{
        \ensuremath{
          {\left(
               \frac{\partial {\ensuremath{#1}}}{\partial {\ensuremath{#2}}}
        \right)}_{\ensuremath{#3}} }
}%
\newcommand{\dd}{\mathrm{d}}

\newcommand{\Derivdd}[2]{
        \ensuremath{\frac{\dd {#1}}{\dd {#2}}}
}%
\newcommand{\Deriv}[2]{\ensuremath{\cfrac{\dd{#1}}{\dd{#2}}}}%BB



\newcommand{\Bnabla}{\ensuremath{\boldsymbol{\nabla}}}


%------------ new (not in rmb_mathTypesetting.sty
\newcommand{\aprx}[1]{\tilde{#1}}%
\newcommand{\Array}[1]{\ensuremath{\hat{\bm{#1}}}\xspace}%
\newcommand{\AArray}[1]{\ensuremath{\hat{\hat{\bm{#1}}}}\xspace}%
\newcommand{\del}{\Bnabla}


%%%%%%%%%%%%%%%%%%%%%%%%%%%%%%%%%%%%%%%%%%%%%%%%%%%%%%%% From rmb_nomenclature.sty
\newcommand{\oneD}{1-D\xspace}
\newcommand{\twoD}{2-D\xspace}
\newcommand{\threeD}{3-D\xspace}
\newcommand{\x}{\Vector{x}}
\newcommand{\Time}{\Scalar{t}}
%------------ new (not in rmb_nomenclature)
\newcommand{\NBF}{\Scalar{N}}%nodal basis function (also nodal shape function)
\newcommand{\NBFa}{\Array{N}}%nodal basis array
\newcommand{\NBFip}{\Scalar{\phi_{ip}}}%avg of \NBF_i over the \pth particle
\newcommand{\gradNBFip}{\Scalar{\Vector{G}_{ip}}}%avg of \NBF_i over the \pth particle
\newcommand{\gradNBF}{\Vector{G}}
\newcommand{\gradNBFa}{\AArray{G}}%defined in each section
\newcommand{\GBF}{\Scalar{\phi}}%grid basis function (also nodal shape function)
\newcommand{\numnodes}{\ensuremath{{\nu_\text{n}}}\xspace}%
\newcommand{\numelements}{\ensuremath{{\nu_\text{e}}}\xspace}%
\newcommand{\numgaussTotal}{\ensuremath{{\nu_\text{G}}}\xspace}% number of gauss points TOTAL
\newcommand{\numgaussOnElemente}{\ensuremath{{\nu_\text{g}(e)}}\xspace}%
\newcommand{\manuscript}{manuscript\xspace}
\newcommand{\ibase}{\Vector{i}}
\newcommand{\jbase}{\Vector{j}}
\newcommand{\kbase}{\Vector{k}}
%%%%%%%%%%%%%%%%%%%%%%%%%%%%%%%%%%%%%%%%%%%%%%%%%%%%%%%% From rmb_paragraphStyles.sty
%\renewcommand{\qedsymbol}{\ensuremath{\square}}
\providecommand{\qed}{........................................................}
\newcounter{example}[chapter]
  \newenvironment{example}[1][]{
    \refstepcounter{example}
    {\setlength{\leftmargin}{9cm}}%
    \subsubsection{EXAMPLE \thechapter.\arabic{example} #1}%
%\sf
    }
    {
\mbox{}
     \\
     \qed \\
    }

  \renewcommand{\theexample}{\thechapter.\arabic{example}}

\newcommand{\Bal}{\begin{aligned}}
\newcommand{\Eal}{\end{aligned}}



%%%%%%%%%%%%%%%%%%%%%%%%%%%%%%%%%%%%%%%%%%%%%%%%%%%%%%% From rmb_textStyles.sty
\definecolor{DefnColor}{rgb}{.1,.1,.6}
\newcommand{\Defn}[1]{{\color{DefnColor} {\sffamily{\textbf{#1}}}}}%
\newcommand{\eg}{\textit{e.g.,~}}
\newcommand{\ie}{\textit{i.e.,~}}
\newcommand{\cf}{\textit{cf.,~}}
\definecolor{Code}{rgb}{0,0,.6}
\newcommand{\code}[1]{{\color{Code} {\textup{\texttt{#1}}}}}
\newcommand{\etal}{\textit{et al.}\xspace}

%%%%%%%%%%%%%%%%%%%%%%%%%%%%%%%%%%%%%%%%%% From ControllingOverbarWidth.tex tutorial
\makeatletter
\newsavebox\myboxA
\newsavebox\myboxB
\newlength\mylenA
\newcommand*\xoverline[2][0.75]{%
    \sbox{\myboxA}{$\m@th#2$}%
    \setbox\myboxB\null% Phantom box
    \ht\myboxB=\ht\myboxA%
    \dp\myboxB=\dp\myboxA%
    \wd\myboxB=#1\wd\myboxA% Scale phantom
    \sbox\myboxB{$\m@th\overline{\copy\myboxB}$}%  Overlined phantom
    \setlength\mylenA{\the\wd\myboxA}%   calc width diff
    \addtolength\mylenA{-\the\wd\myboxB}%
    \ifdim\wd\myboxB<\wd\myboxA%
       \rlap{\hskip 0.5\mylenA\usebox\myboxB}{\usebox\myboxA}%
    \else
        \hskip -0.5\mylenA\rlap{\usebox\myboxA}{\hskip 0.5\mylenA\usebox\myboxB}%
    \fi}
\makeatother




%%%%%%%%%%%%%%%%%%%%%%%%%%%%%%%%%%%%%%%%%%%%%%%%%%%%%%%%%% unfiled
\newcommand{\alt}{\text{*}}
\newcommand{\gimpW}{\ensuremath{\omega}\xspace}




